%% -*- Mode: LaTeX -*-
%%
%% template.tex
%% Created Wed Jul 25 09:32:40 AKDT 2007
%% Last updated Wed 16 Mar 2022 11:37:56 AM AKDT
%% by Raymond E. Marcil <marcilr@gmail.com>
%% 
%% LaTeX Template with Examples
%%

  %%
%%%%%% Preamble.
  %%

%% Specify DVIPS driver used by things like hyperref
\documentclass[12pt,letterpaper,dvips]{article}



%%
%% Adding unnumbered sections to TOC
%% Asked 11 years ago, Modified 4 months ago, Viewed 101k times
%%
%% The easiest way to do this is to use \section but to change
%% secnumdepth so that they don't get numbered.
%%
%% \documentclass{article}
%% \setcounter{secnumdepth}{0}
%% \begin{document}
%% \tableofcontents
%% \section{ADSF}
%% \section{Foo}
%% \end{document}
%% \setcounter{secnumdepth}{0}
%%
%% Adding unnumbered sections to TOC
%% Asked 11 years ago, Modified 4 months ago, Viewed 101k times
%% https://tex.stackexchange.com/questions/11668/adding-unnumbered-sections-to-toc
%%
\setcounter{secnumdepth}{0}

%% Displays nice fully qualified date
\usepackage{datetime}

%%
%% Make for pretty date like:
%% Monday 4th February, 2013
%% datetime.sty v2.60: Formatting
%% Current Date and Time
%% Nicola L. C. Talbot
%% Dickimaw Books
%% http://www.dickimaw-books.com/
%% 2015-03-20
%% https://ctan.math.washington.edu/tex-archive/obsolete/macros/latex/contrib/datetime/datetime.pdf
%%
%%\longdate


%% rcs is the package to display cvs revision info.
%%\usepackage{rcs}
\usepackage{fullpage}
\usepackage{fancyvrb} 
\usepackage{graphicx}
\usepackage{figsize}
\usepackage{calc}
\usepackage{fancyhdr}
\pagestyle{fancy}

%%
%% enumitem – Control layout of itemize, enumerate, description
%% https://www.ctan.org/pkg/enumitem
%%
%% Allows for use of \bgein{itemize}[leftmargin=0pt] 
%% to lists with 0 left margin.
%%
%% Itemize left margin
%% http://tex.stackexchange.com/questions/170525/itemize-left-margin
%% 
\usepackage{enumitem}%     http://ctan.org/pkg/enumitem


%% caption package for use in justifying table or figure captions
\usepackage{caption}

\usepackage{xspace}
\usepackage{booktabs}
\usepackage[first,bottomafter]{draftcopy}
\usepackage[numbib]{tocbibind}
\usepackage{amssymb}              %% AMS Symbols, used for \checkmark
\usepackage{multicol}

%%
%% Extract SVN metadata for use elsewhere.
%% This information has:
%% o the filename
%% o the revision number
%% o the date and time of the last Subversion co command
%% o name of the user who has done the action
%%
%%\usepackage{svninfo}
%%\svnInfo $Id: template.tex 52 2013-02-04 22:32:54Z marcilr $

%%
%% Hyperref package for embedding URLs for clickable links in PDFs, 
%% also specify PDF attributes here.
%%
%% The pdfborder={0 0 0} is what ellimated the blue box around the url
%% displayed by \href{}{}.
%%
%% The command pdfborder={0 0 1} would display a box with thickness of 1 pt.
%%
%% Hypertext marks in LATEX: a manual for hyperref
%% by Sebastian Rahtz and Heiko Oberdiek - November 2012
%% http://ctan.org/pkg/hyperref 
%% http://mirror.hmc.edu/ctan/macros/latex/contrib/hyperref/doc/manual.html
%%
\usepackage[
colorlinks,
linkcolor=blue,
%%colorlinks=false,
hyperindex=false,
urlcolor=blue,
pdfborder={0 0 0},
pdfauthor={Raymond E. Marcil},
pdftitle={LaTeX Template with Examples},
pdfcreator={ps2pdf},
pdfsubject={LaTeX Examples},
pdfkeywords={LaTeX Examples}
]{hyperref}


%%
%% Extract RCS metadata for use elsewhere.
%% Jason figured this out, very cool.
%%
%%\RCS $Revision: 1.53 $
%%\RCS $Date: 2006/06/26 21:04:55 $


  %%
%%%%%% Customization.
  %%

% On letter paper with 10pt font the Verbatim environment has 65 columns.
% With 12pt font the environment has 62 columns.  Exceeding this will exceed
% the frame and will look ugly.  YHBW.  HAND.
\RecustomVerbatimEnvironment{Verbatim}{Verbatim}{frame=single}

\renewenvironment{description}
                 {\list{}{\labelwidth 0pt \iteminden-\leftmargin
                          \let\labelsep\hsize
                          \let\makelabel\descriptionlabel}}
                 {\endlist}
\renewcommand*\descriptionlabel[1]{\hspace\labelsep\sffamily\bfseries #1}


  %%
%%%%%% Commands.
  %%

\newcommand{\FIXME}[1]{\textsf{[FIXME: #1]}}
\newcommand{\cmd}[1]{\texttt{#1}}


%% Squeeze space above/below captions
\setlength{\abovecaptionskip}{4pt}   % 0.5cm as an example
\setlength{\belowcaptionskip}{4pt}   % 0.5cm as an example


%% Tex really adds a lot of whitespace to itemized 
%% lists so define a new command itemize* with a 
%% lot less whitespace.  Found this in the British
%% Tex faq.
\newenvironment{itemize*}%
  {\begin{itemize}%
    \setlength{\itemsep}{0pt}%
    \setlength{\parsep}{0pt}}%
  {\end{itemize}}

  
%%
%% Tex really adds a lot of whitespace to itemized 
%% lists so define a new command itemize* with a 
%% lot less whitespace.  Found this in the British
%% Tex faq.
%%
%% Tue Jun 23 13:22:04 AKDT 2015
%% =============================
%% Added [leftmargin=0.0mm] to set the left margin=0
%% This requires use of the enumitem package:
%%   \usepackage{enumitem}%     http://ctan.org/pkg/enumitem
%%
%% Itemize left margin
%% http://tex.stackexchange.com/questions/170525/itemize-left-margin
%%
\newenvironment{itemizenoleft*}%
  {\begin{itemize}[leftmargin=15.0pt]%
    \setlength{\itemsep}{0pt}%
    \setlength{\parsep}{0pt}}%
  {\end{itemize}}
  

%%
%% Tex really adds a lot of whitespace to itemized 
%% lists so define a new command enumerate* with a 
%% lot less whitespace.  Created using itemize*
%% pattern.  
%%
  \newenvironment{enumerate*}%
  {\begin{enumerate}%
    \setlength{\itemsep}{0pt}%
    \setlength{\parsep}{0pt}}%
  {\end{enumerate}}


%%
%% Tex really adds a lot of whitespace to itemized 
%% lists so define a new command enumerate* with a 
%% lot less whitespace.  Created using itemize*
%% pattern.  
%%
%% Tue Jun 23 13:22:04 AKDT 2015
%% =============================
%% Added [leftmargin=0.0mm] to set the left margin=0
%% This requires use of the enumitem package:
%%   \usepackage{enumitem}%     http://ctan.org/pkg/enumitem
%%
%% Itemize left margin
%% http://tex.stackexchange.com/questions/170525/itemize-left-margin
%%
\newenvironment{enumeratenoleft*}%
  {\begin{enumerate}[leftmargin=0.0mm]%
    \setlength{\itemsep}{0pt}%
    \setlength{\parsep}{0pt}}%
  {\end{enumerate}}


%% Squeeze space
\renewcommand\floatpagefraction{.9}
\renewcommand\topfraction{.9}
\renewcommand\bottomfraction{.9}
\renewcommand\textfraction{.1}   
\setcounter{totalnumber}{50}
\setcounter{topnumber}{50}
\setcounter{bottomnumber}{50}

%%
%% Used for inline space using \hard
%% Using betweem date and build time
%%
%% Special spaces in LaTeX
%% Asked 9 years, 2 months ago
%% Modified 9 years, 2 months ago
%% Viewed 2k times
%% --answered Jan 3, 2013 at 14:12, egreg
%% https://tex.stackexchange.com/questions/89075/special-spaces-in-latex
\newcommand{\hard}[1]{\unskip\hspace{#1}\ignorespaces}


  %%
%%%%%% Document.
  %%

\title{\LaTeX\ Template}

\author{Raymond E. Marcil\\
        \texttt{$<$marcilr@gmail.com$>$}
}


\begin{document}

%%
%% This displays date like:
%%   Wednesday 16th March, 2022 19:13 
%%
%% Note the build time appended the date.
%% Very nice!
%%
%% \hard defined above with:
%%   \newcommand{\hard}[1]{\unskip\hspace{#1}\ignorespaces}
%%
%% Special spaces in LaTeX
%% Ask Question
%% Asked 9 years, 2 months ago
%% Modified 9 years, 2 months ago
%% Viewed 2k times
%% https://tex.stackexchange.com/questions/89075/special-spaces-in-latex
%%
\date{\today\hard{0.3em}\currenttime}

\maketitle

%% 
%% \maketitle command. As by default, it executes in the hidden
%% command \date{\today}.
%% Add current time to \today with package datetime? (LaTeX)
%% Asked 11 years, 4 months ago
%% Modified 8 years, 1 month ago
%% Viewed 56k times
%% asked Oct 31, 2010 at 18:16, vy32
%% edited Jan 23, 2014 at 8:57, strpeter
%%
%% This displays date like:
%% Wednesday 16th March, 2022
%%
%%\date{\today}


%% ====================== Abstract =====================
%% ====================== Abstract =====================
%% abstract.tex
%% Created Wed 16 Mar 2022 11:39:37 AM AKDT
%% Copyright (C) 2022 by Raymond E. Marcil <marcilr@gmail.com>
%%

%% ===================== Abstract ======================
%% ===================== Abstract ======================
\begin{abstract}
     This document contains \LaTeX\ Template with Examples.

\end{abstract}


\vspace{2.0in}

%% Draw DNR logo and address at bottom of page
%%\begin{figure}[h]
%%        \hspace{0.32in}
%%        \SetFigLayout{1}{1}
%%        \begin{minipage}[b]{0.16\figwidth}
%%                \includegraphics[width=\textwidth]{dnr_bwlogo.eps}
%%        \end{minipage}
%%        \hspace{5pt}
%%        \begin{minipage}[b]{\figwidth}
%%                \bf{Alaska Department of Natural Resources}\\
%%                \small{\sf{Division of Support Services\\
%%                Land Records Information Section\\
%%                550 W. 7th Ave. Suite 706\\
%%                Anchorage, Alaska 99501}}
%%        \end{minipage}
%%\end{figure}

\newpage
\tableofcontents

\newpage
\listoffigures
\listoftables

%% ====================== Sections =====================
%% ====================== Sections =====================
%% abbreviations.tex
%% Created Fri Jun 14 07:43:18 AKDT 2019
%% Copyright (C) 2019 by Raymond E. Marcil <marcilr@gmail.com>
%%

%% =============== List of Abbreviations ===============
%% =============== List of Abbreviations ===============
\newpage
\setcounter{secnumdepth}{0}
\section{List of Definitions and Abbreviations}
\begin{itemize*}
  \item{\begin{bf}MOA\end{bf}} - Municipality of Anchorage

\end{itemize*}
       %% List of Abbreviations
%% preface.tex
%% Created Wed 16 Mar 2022 03:39:21 PM AKDT
%% Copyright (C) 2022 by Raymond E. Marcil <marcilr@gmail.com>
%%

%% ======================== Preface ==============================
%% ======================== Preface ==============================
\newpage
\section{Preface}

Preface...
             %% Preface
%% introduction.tex
%% Created Wed 16 Mar 2022 11:47:44 AM AKDT
%% Copyright (C) 2022 by Raymond E. Marcil <marcilr@gmail.com>
%%

%% ====================== Introduction ===========================
%% ====================== Introduction ===========================
\newpage
\section{Introduction}
Introduction to the \LaTeX\ Template with Examples.
        %% Introduction
%% examples.tex
%% Created Wed 16 Mar 2022 12:00:19 PM AKDT
%% Copyright (C) 2022 by Raymond E. Marcil <marcilr@gmail.com>
%%

%% ======================== Examples =============================
%% ======================== Examples =============================
\newpage
\section{Examples}
Series of useful \LaTeX\ markup. Need to break out to 
separate examples.tex file.

\subsection{Escaping $<$ and $>$ Symbols}
To get \$$<$\$ or \$$>$\$ just wrap the symbols in \$ for math mode.

\subsection{Enumerate}
\begin{enumerate}
  \item{DNR} - Alaska State Department of Natural Resources
    \begin{itemize*}
      \item{HI} - Historical Index, not maintained since 1982
      \item{LE} - Land Estate, maintained by SGU
      \item{ME} - Mineral Estate, maintaind by SGU
    \end{itemize*}

  \item{Alaska State Surveys}
    \begin{itemize*}
      \item{ASBLT} - As-Built Survey
      \item{ASCS} - Cadastral Survey
    \end{itemize*}
\end{enumerate}
   %% Examples
%% caption.tex
%% Created Wed 16 Mar 2022 02:04:16 PM AKDT
%% Copyright (C) 2022 by Raymond E. Marcil <marcilr@gmail.com>
%%

%% ===================== Caption ======================
%% ===================== Caption ======================
%%
%% Configure caption on left using caption package.
%% The key to this working is placing the \captionsetup
%% just before the table or figure for which to manipulate
%% the package.
%%
%% Other potential potential options include:
%%   labelsep=newline
%%   labelfont=bf
%%
%% Links
%% =====
%% Raggedright and caption package
%% http://tex.stackexchange.com/questions/120411/raggedright-and-caption-package
%%
\captionsetup{%
    justification=raggedright,
    singlelinecheck=false
}

\noindent\FIXME{Need to complete details here}
             %% Caption
%% questions.tex
%% Created Wed 16 Mar 2022 11:55:08 AM AKDT
%% Copyright (C) 2022 by Raymond E. Marcil <marcilr@gmail.com>
%%

%% ====================== Questions ====================
%% ====================== Questions ====================
\section{Questions}
\begin{itemize*}
     \item{} Break table column over two rows?
\end{itemize*}
           %% Questions
%% endnotes.tex
%% Created Wed 16 Mar 2022 01:49:23 PM AKDT
%% Copyright (C) 2022 by Raymond E. Marcil <marcilr@gmail.com>
%%

%% ======================== Endnotes =============================
%% ======================== Endnotes =============================
\clearpage
\newpage
\setcounter{secnumdepth}{0}
\section{Endnotes}

\begin{enumerate}

%% ====================== LaTeX Verbatim =========================
\item \LaTeX\ verbatim\\
\href{http://www.kfunigraz.ac.at/~binder/texhelp/ltx-79.html}
{http://www.kfunigraz.ac.at/~binder/texhelp/ltx-79.html}

\end{enumerate}

            %% Endnotes
%% appendix.tex
%% Created Wed 16 Mar 2022 01:53:48 PM AKDT
%% Copyright (C) 2022 by Raymond E. Marcil <marcilr@gmail.com>
%%

%% ======================== Appendix =============================
%% ======================== Appendix =============================
%%
%% This will add a standard non-numbered Appendix to the document.
%% The next section Appendix chnages secnumdepth such that the Appendix
%% is not numbered but still displayed in the table of contents (TOC).
%%
%% Adding unnumbered sections to TOC
%% http://tex.stackexchange.com/questions/11668/adding-unnumbered-sections-to-toc
%% 
%% \section*{Appendix}
%% Need content here.
%%
\setcounter{secnumdepth}{0}
\section{Appendix}


%% ============= LaTeX - Wikibooks ===============
A Guide to \LaTeX\\\
\href{http://www.astro.rug.nl/~kuijken/latex.html}
{http://www.astro.rug.nl/~kuijken/latex.html}
\\
\\
\LaTeX\ - From Wikibooks,the open-content textbooks collection\\
\href{http://en.wikibooks.org/wiki/LaTeX}{http://en.wikibooks.org/wiki/LaTeX}
\\
\\
\LaTeX\ Notes\\
\href{http://luke.breuer.com/time/item/LaTeX\_Notes/180.aspx}{http://luke.breuer.com/time/item/LaTeX\_Notes/180.aspx}
            %% Appendix

\end{document}

%% Local Variables:
%% fill-column: 78
%% mode: auto-fill
%% compile-command: "make"
%% End:
