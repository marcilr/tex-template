%% -*- Mode: LaTeX -*-
%%
%% latex-examples.tex
%% Created Tue Feb 23 16:38:43 AKST 2010
%% by Raymond E. Marcil <marcilr@gmail.com>
%% Copyright (C) 2010 Raymond E. Marcil
%%
%% Permission is granted to copy, distribute and/or modify this document
%% under the terms of the GNU Free Documentation License, Version 1.3
%% or any later version published by the Free Software Foundation;
%% with no Invariant Sections, no Front-Cover Texts, and no Back-Cover Texts.
%% A copy of the license is included in the section entitled "GNU
%% Free Documentation License".
%%


%%

  %%
%%%%%% Preamble.
  %%

%% Specify DVIPS driver used by things like hyperref
\documentclass[12pt,letterpaper,dvips]{article}

%% Embedded cvs version number not needed.
%% (rem - Feb 22, 2010)
%%\usepackage{rcs}

\usepackage{fullpage}
\usepackage{fancyvrb} 
\usepackage{graphicx}
\usepackage{figsize}
\usepackage{calc}
\usepackage{xspace}
\usepackage{booktabs}
\usepackage[first,bottomafter]{draftcopy}
\usepackage[numbib]{tocbibind}
\usepackage{amssymb}              %% AMS Symbols, used for \checkmark
\usepackage{multicol}


%% Hyperref package for embedding URLs
%% for clickable links in PDFs, 
%% also specify PDF attributes here.
\usepackage[
colorlinks=false,
hyperindex=false,
urlcolor=blue,
pdfauthor={Raymond E. Marcil},
pdftitle={LaTeX Examples},
pdfcreator={ps2pdf},
pdfsubject={LaTeX Examples},
pdfkeywords={LaTeX, examples, syntax}
]{hyperref}


  %%
%%%%%% Customization.
  %%

% On letter paper with 10pt font the Verbatim environment has 65 columns.
% With 12pt font the environment has 62 columns.  Exceeding this will exceed
% the frame and will look ugly.  YHBW.  HAND.
\RecustomVerbatimEnvironment{Verbatim}{Verbatim}{frame=single}

\renewenvironment{description}
                 {\list{}{\labelwidth 0pt \iteminden-\leftmargin
                          \let\labelsep\hsize
                          \let\makelabel\descriptionlabel}}
                 {\endlist}
\renewcommand*\descriptionlabel[1]{\hspace\labelsep\sffamily\bfseries #1}


  %%
%%%%%% Commands.
  %%

\newcommand{\FIXME}[1]{\textsf{[FIXME: #1]}}
\newcommand{\cmd}[1]{\texttt{#1}}

%% Squeeze space above/below captions
\setlength{\abovecaptionskip}{4pt}   % 0.5cm as an example
\setlength{\belowcaptionskip}{4pt}   % 0.5cm as an example


%% Tex really adds a lot of whitespace to itemized 
%% lists so define a new command itemize* with a 
%% lot less whitespace.  Found this in the British
%% Tex faq.
\newenvironment{itemize*}%
  {\begin{itemize}%
    \setlength{\itemsep}{0pt}%
    \setlength{\parsep}{0pt}}%
  {\end{itemize}}

%% Squeeze space
\renewcommand\floatpagefraction{.9}
\renewcommand\topfraction{.9}
\renewcommand\bottomfraction{.9}
\renewcommand\textfraction{.1}   
\setcounter{totalnumber}{50}
\setcounter{topnumber}{50}
\setcounter{bottomnumber}{50}


  %%
%%%%%% Document.
  %%

\title{\LaTeX\ Examples}

\author{Raymond E. Marcil\\
        \texttt{marcilr@gmail.com}\\
        Version: 0.0.4
}

\begin{document}

\maketitle

\begin{abstract}
     This document contains \LaTeX\ examples.

\end{abstract}

\vspace{4.0in}

%% Draw UAA ITS logo and address at bottom of page.
%%\begin{figure}[h]
%%        \hspace{0.32in}
%%        \SetFigLayout{1}{1}
%%        \begin{minipage}[b]{0.16\figwidth}
%%                \includegraphics[width=\textwidth]{dnr_bwlogo.eps}
%%        \end{minipage}
%%        \hspace{5pt}
%%        \begin{minipage}[b]{\figwidth}
%%                \bf{IT Services}\\
%%                \small{\sf{University of Alaska Anchorage\\
%%                3211 Providence Drive\\
%%                Anchorage, Alaska 99508\\
%%                callcenter@uaa.alaska.edu\\
%%                (907)786-4646}}
%%        \end{minipage}
%%\end{figure}

\newpage


\tableofcontents

\newpage
\listoffigures
\listoftables


\newpage

%% =============== List of Abbreviations ===============
%% =============== List of Abbreviations ===============
\section{List of Definitions and Abbreviations}
\begin{itemize*}
  \item{\begin{bf}MOA\end{bf}} - Municipality of Anchorage

\end{itemize*}

%% =================== Introduction ====================
%% =================== Introduction ====================

\newpage
\section{Introduction}
The bsdInspect web application provides Inspection Requests
submission.


%% ======================== Questions ============================
%% ======================== Questions ============================
\section{Questions}
Still have some outstanding questions.

\begin{itemize*}
     \item{} - Break table column over two rows?
\end{itemize*}

%% ======================== Examples =============================
%% ======================== Examples =============================
%% ======================== Examples =============================
\newpage
\section{Examples}
Series of useful \LaTeX\ markup. Need to break out to 
separate examples.tex file.

\subsection{Escaping $<$ and $>$ Symbols}
To get \$$<$\$ or \$$>$\$ just wrap the symbols in \$ for math mode.

\subsection{Enumerate}
\begin{enumerate}
  \item{DNR} - Alaska State Department of Natural Resources
    \begin{itemize*}
      \item{HI} - Historical Index, not maintained since 1982
      \item{LE} - Land Estate, maintained by SGU
      \item{ME} - Mineral Estate, maintaind by SGU
   \end{itemize*}

  \item{Alaska State Surveys}
    \begin{itemize*}
      \item{ASBLT} - As-Built Survey
      \item{ASCS} - Cadastral Survey
    \end{itemize*}
\end{enumerate}


%% ======================= Comments =======================
%% ======================= Comments =======================
\subsection{Comments}
\begin{center}
\framebox{
\begin{minipage}[t]{0.9\textwidth}
\cmd{COMMENTS} Comment --- \emph{Sean Weems, Spring 2003}\\
We should get the \cmd{COMMENTS} column searchable via the 
landrecords application before we do much anything else -- shouldn't
be too hard.
\end{minipage}
}
\end{center}

\begin{center}
\framebox{
\begin{minipage}[t]{0.9\textwidth}
\emph{Errata: Plats spanning multiple sections}\\
A few anomalies can be observed in the \cmd{AKPLATS}
table. Specifically plats exist that span multiple sections. 
Since the table only has a single column, \cmd{SCODE}, 
that accepts a single section code, SGU (Status
Graphics Unit) has handled this problem by entering multiple 
rows in the table, each with a different section that point to
the same plat or file. Multiple section plats are indicated by  setting 
the \cmd{TCODE} column to the value \cmd{37}, and making an 
appropriate notation like \emph{Section 24-25-26-27} in the 
\cmd{REMARKS} column.\\
\FIXME{Perhaps the \cmd{SCODE} column should accept an array of sections?}

\end{minipage}
}
\end{center}

%% ====================== Footnotes ======================
%% ====================== Footnotes ======================
\subsection{Footnotes}
 Some footnotes here
\ref{fig:uss-index} for an example. Yet another
\ref{table:uss-index} example.

%% ================ Simple Table Examples ================
%% ================ Simple Table Examples ================
\newpage
\subsection{Simple Table Examples}
\begin{table}[htb]
\begin{tabular}{|p{.25\textwidth}|p{.20\textwidth}|p{.47\textwidth}|}\hline 
Column Name&Type&Description\\ 
\hline
EQS&VARCHAR2(1)&!NULL map shows village selections\\
ITM\_COL&VARCHAR2(1)&USGS ITM column: 1-6\\
ITM\_ROW&VARCHAR2(1)&USGS ITM row: A-E\\
QMQ\_ABBR\_DNR&VARCHAR2(3)&Three character DNR abbreviation for the QMQ\\
RASTER\_FILENAME&VARCHAR2(50)&Physical path to file\\
RASTER\_PATHNAME&VARCHAR2(50)&URL path to PDF of map\\
SCODE&VARCHAR2(2)&Supplement map code: 1,2,3,...\\
COMMENTS&VARCHAR2(256)&Plat comments\\
\hline
\end{tabular}
\caption {\cmd{EASEMENTS\_17B} Table}
\label{table:easements_17b}
\end{table}

%% A simple table example
%% The htb attribute attempts to inline the table or figure
%% where you put it in the document
\begin{table}[htb]
\begin{center}
\begin{tabular*}{\textwidth}{@{}p{.25\textwidth}@{}p{.75\textwidth}}
\hline
\hline\\[-2.5ex]
XML element&Descripton\\
\hline
\hline\\[-1.5ex]   %% Trick to add whitespace after horizontal line
FNUM&US Survey file number\\ 
MERIDIAN&BLM meridian code\\
&\hspace{10pt}12 = Copper River\\
&\hspace{10pt}13 = Fairbanks\\
&\hspace{10pt}28 = Seward\\
&\hspace{10pt}44 = Kateel\\
&\hspace{10pt}45 = Umiat\\
TOWNSHIP&Five character Township code\\
RANGE&Five character Range code\\
PAGE&Survey page number 1,2,3,...\\
FILENAME&Relative path to file in direcory\\[1.5ex]
\hline
\end{tabular*}
\caption {USS XML index elements}
\label{table:uss-index}
\end{center}
\end{table}


%%
%% NOTE: tabular* takes table width argument
%%       Use of 'p' for left aligned column.
%%       Use of 'cp' to center column.
%%       The @{} works to remove an unwanted space.
%%
%% FIXME: Set width of centered columns?
%%
%% Links
%% =====
%% Getting to Grips with Latex - Tables
%% by Andrew Roberts
%% Source for @{\extracolsep{\fill} syntax.
%% http://www.andy-roberts.net/misc/latex/latextutorial4.html
%%
\begin{table}[htb]
\begin{center}
\begin{tabular*}{0.90\textwidth}{@{\extracolsep{\fill}}@{}l@{}c@{}c@{}c}
  \hline
  \hline\\[-2.5ex]
  col 1 & col 2 & col 3 & col 4 \\
  \hline
  \hline\\[-1.5ex]
  item 1 & item 2 & item 3 & item 4 \\
  \hline 
  item 1  & item 2  & item 3  & item 4  \\
  \hline
\end{tabular*}
\caption {Demo}
\label{table:demo-index}
\end{center}
\end{table}

%% ================== Installed Daemons =================== 
%% The htb attribute attempts to inline the table or figure
%% where you put it in the document
%%
%% NOTE: tabular* takes table width argument
%%       Use of 'p' for left aligned column.
%%       Use of 'cp' to center column.
%%       The @{} works to remove an unwanted space.
%%
%% FIXME: The ELM adn Elluminate Server columns are not centering.
%%        Center with fixed width in LaTeX?
%%
%% Links
%% =====
%% Getting to Grips with Latex - Tables
%% by Andrew Roberts
%% http://www.andy-roberts.net/misc/latex/latextutorial4.html
%%
\begin{table}[htb]
\begin{center}
\begin{tabular*}{.80\textwidth}{@{\extracolsep{\fill}}@{}p{.25\textwidth}@{}c@{}c@{}c@{}c}
\hline
\hline\\[-2.5ex]
Virtual Machine&Apache&ELM&LM&Elluminate Server\\
\hline
\hline\\[-1.5ex]   %% Trick to add whitespace after horizontal line
\cmd{dcs-elive-prod01}&\,&x&x&x\\
\cmd{uaa-elive-dev01}&x&x&x&\\
\cmd{uaa-elive-server01}&\,&\,&\,&x\\[1.5ex]
\cmd{uaa-elive-prod01}&\,&x&x&x\\
\cmd{uaf-elive-prod01}&\,&x&x&x\\
\cmd{uas-elive-prod01}&\,&x&x&x\\
\hline
\end{tabular*}
\caption {Daemons}
\label{table:daemons-index}
\end{center}
\end{table}

\FIXME{Center with fixed width in LaTeX?}


\begin{table}[htb]
\begin{tabular}{|p{.16\textwidth}|p{.17\textwidth}|p{.62\textwidth}|}\hline 
Column Name&Type&Description\\ 
\hline
MTR&VARCHAR2(9)&Meridian, Township, Range, example: \emph{C026S054E}\\
QMQ&VARCHAR2(3)&Quarter Million Quadrangle code,\\
&&\hspace{10pt}example: \emph{DIL} (Dillingham quadrangle)\\
\hline
\end{tabular}
\caption {\cmd{XREF\_MTR\_QMQ} Table}
\label{table:xref_mtr_qmq}
\end{table}




%% ====================== Verbatim ==========================
%% ====================== Verbatim ==========================
\newpage
\subsection{Verbatim}

``The verbatim environment is a paragraph-making environment that gets 
\LaTeX\ to print exactly what you type in. It turns \LaTeX\ into a typewriter 
with carriage returns and blanks having the same effect that they would 
on a typewriter.''\footnotemark

\begin{verbatim}
\begin{verbatim}
    text
\end{verbatim\}
\end{verbatim}

\FIXME{Need to determine how to escape the verbatim markup for display above.}

\begin{center}
\begin{figure}
\begin{quote}
\small
\begin{Verbatim}[frame=none]
gis/raster/
  dnr/
    map_library/
    plats/
      SP/YYYYMMDD/*.pdf               # indexed
      HI/YYYYMMDD/*.pdf               # Indexed
      ASLS/YYYYMMDD/*.pdf             # Indexed
    recorded-plats/
      YYYYMMDD/*.pdf
  blm/
    easements_17b/YYYYMMDD/*.pdf      # indexed
    mtp/YYYYMMDD/*.pdf                # non-indexed
    usrs/YYYYMMDD/*.pdf               # indexed
    usrs-notes/YYYYMMDD/*.pdf         # indexed
    uss/YYYYMMDD/*.pdf                # indexed
    uss-notes/YYYYMMDD/*.pdf          # indexed
    usms/YYYYMMDD/*.pdf               # indexed
    usms-notes/YYYYMMDD/*.pdf         # indexed
  usgs/
    drg/
      collared/ 
        250K/
        63K/
        25K/
        24/
      decollared/
      tools/
      missing\_data/
    dem/
    doq/
    topo/
\end{Verbatim}
\normalsize
\end{quote}
\caption{Verbatim Example}
\label{fig:verbatimexample}
\end{figure}
\end{center}




%% ======================== Endnotes =============================
%% ======================== Endnotes =============================
\clearpage
\newpage
\section*{Endnotes}

\begin{enumerate}

%% ====================== LaTeX Verbatim =========================
\item \LaTeX\ verbatim\\
\href{http://www.kfunigraz.ac.at/~binder/texhelp/ltx-79.html}
{http://www.kfunigraz.ac.at/~binder/texhelp/ltx-79.html}

\end{enumerate}


%% ======================== Appendix =============================
%% ======================== Appendix =============================
\clearpage
\newpage
\section*{Appendix}


%% ============= LaTeX - Wikibooks ===============
A Guide to \LaTeX\\\
\href{http://www.astro.rug.nl/~kuijken/latex.html}
{http://www.astro.rug.nl/~kuijken/latex.html}
\\
\\
\LaTeX\ - From Wikibooks,the open-content textbooks collection\\
\href{http://en.wikibooks.org/wiki/LaTeX}{http://en.wikibooks.org/wiki/LaTeX}
\\
\\
\LaTeX\ Notes\\
\href{http://luke.breuer.com/time/item/LaTeX\_Notes/180.aspx}{http://luke.breuer.com/time/item/LaTeX\_Notes/180.aspx}

\end{document}

%% Local Variables:
%% fill-column: 78
%% mode: auto-fill
%% compile-command: "make"
%% End:

