%% table_examples.tex
%% Created Wed 16 Mar 2022 12:58:43 PM AKDT
%% Copyright (C) 2022 by Raymond E. Marcil <marcilr@gmail.com>
%%

%% ================ Table Examples ================
%% ================ Table Examples ================
\subsection{Table Examples}
\begin{table}[htb]
\begin{tabular}{|p{.25\textwidth}|p{.20\textwidth}|p{.47\textwidth}|}\hline 
Column Name&Type&Description\\ 
\hline
EQS&VARCHAR2(1)&!NULL map shows village selections\\
ITM\_COL&VARCHAR2(1)&USGS ITM column: 1-6\\
ITM\_ROW&VARCHAR2(1)&USGS ITM row: A-E\\
QMQ\_ABBR\_DNR&VARCHAR2(3)&Three character DNR abbreviation for the QMQ\\
RASTER\_FILENAME&VARCHAR2(50)&Physical path to file\\
RASTER\_PATHNAME&VARCHAR2(50)&URL path to PDF of map\\
SCODE&VARCHAR2(2)&Supplement map code: 1,2,3,...\\
COMMENTS&VARCHAR2(256)&Plat comments\\
\hline
\end{tabular}
\caption {\cmd{EASEMENTS\_17B} Table}
\label{table:easements_17b}
\end{table}

%% A simple table example
%% The htb attribute attempts to inline the table or figure
%% where you put it in the document
\begin{table}[htb]
\begin{center}
\begin{tabular*}{\textwidth}{@{}p{.25\textwidth}@{}p{.75\textwidth}}
\hline
\hline\\[-2.5ex]
XML element&Descripton\\
\hline
\hline\\[-1.5ex]   %% Trick to add whitespace after horizontal line
FNUM&US Survey file number\\ 
MERIDIAN&BLM meridian code\\
&\hspace{10pt}12 = Copper River\\
&\hspace{10pt}13 = Fairbanks\\
&\hspace{10pt}28 = Seward\\
&\hspace{10pt}44 = Kateel\\
&\hspace{10pt}45 = Umiat\\
TOWNSHIP&Five character Township code\\
RANGE&Five character Range code\\
PAGE&Survey page number 1,2,3,...\\
FILENAME&Relative path to file in direcory\\[1.5ex]
\hline
\end{tabular*}
\caption {USS XML index elements}
\label{table:uss-index}
\end{center}
\end{table}

%%
%% NOTE: tabular* takes table width argument
%%       Use of 'p' for left aligned column.
%%       Use of 'cp' to center column.
%%       The @{} works to remove an unwanted space.
%%
%% FIXME: Set width of centered columns?
%%
%% Links
%% =====
%% Getting to Grips with Latex - Tables
%% by Andrew Roberts
%% Source for @{\extracolsep{\fill} syntax.
%% http://www.andy-roberts.net/misc/latex/latextutorial4.html
%%
\begin{table}[htb]
\begin{center}
\begin{tabular*}{0.90\textwidth}{@{\extracolsep{\fill}}@{}l@{}c@{}c@{}c}
  \hline
  \hline\\[-2.5ex]
  col 1 & col 2 & col 3 & col 4 \\
  \hline
  \hline\\[-1.5ex]
  item 1 & item 2 & item 3 & item 4 \\
  \hline
  item 1  & item 2  & item 3  & item 4  \\
  \hline
\end{tabular*}
\caption {Demo}
\label{table:demo-index}
\end{center}
\end{table}

\begin{table}[htb]
%%\begin{center}
\begin{tabular*}{.80\textwidth}{@{\extracolsep{\fill}}@{}p{.25\textwidth}@{}c@{}c@{}c@{}c}
\hline
\hline\\[-2.5ex]
Virtual Machine&Apache&ELM&LM&Elluminate Server\\
\hline
\hline\\[-1.5ex]   %% Trick to add whitespace after horizontal line
\cmd{dcs-elive-prod01}&\,&x&x&x\\
\cmd{uaa-elive-dev01}&x&x&x&\\
\cmd{uaa-elive-server01}&\,&\,&\,&x\\[1.5ex]
\cmd{uaa-elive-prod01}&\,&x&x&x\\
\cmd{uaf-elive-prod01}&\,&x&x&x\\
\cmd{uas-elive-prod01}&\,&x&x&x\\
\hline
\end{tabular*}
\caption {Daemons}
\label{table:daemons-index}
%%\end{center}
\end{table}


\begin{table}[htb]
\begin{tabular}{|p{.16\textwidth}|p{.17\textwidth}|p{.62\textwidth}|}\hline 
Column Name&Type&Description\\ 
\hline
MTR&VARCHAR2(9)&Meridian, Township, Range, example: \emph{C026S054E}\\
QMQ&VARCHAR2(3)&Quarter Million Quadrangle code,\\
&&\hspace{10pt}example: \emph{DIL} (Dillingham quadrangle)\\
\hline
\end{tabular}
\caption {\cmd{XREF\_MTR\_QMQ} Table}
\label{table:xref_mtr_qmq}
\end{table}
